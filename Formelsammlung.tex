\documentclass[10pt,a4paper,landscape]{article}
    \usepackage[left=0.55cm,right=0.55cm,top=1.10cm,bottom=0.55cm,landscape,
    headsep=2mm]{geometry}
    
    \usepackage{lastpage}
    \usepackage{fancyhdr}
    \usepackage{multicol}
    \usepackage[utf8]{inputenc}
    \usepackage[ngerman]{babel}
    \usepackage[T1]{fontenc}
    \usepackage{listings}
    \usepackage{enumitem}
    \setitemize{leftmargin=15pt}  
    \setenumerate{leftmargin=15pt}  
    \usepackage{titlesec}
    \usepackage{color,soul}
    \usepackage{graphicx}
    \usepackage{tabularx}
    \usepackage{tikz}
    \usetikzlibrary{automata,positioning}
    \usepackage[babel,german=quotes]{csquotes}
    \usepackage{arydshln}
    \usepackage[fleqn]{amsmath}
    \usepackage{setspace}
    \usepackage{amssymb}
    \usepackage{float}
    \usepackage{booktabs}
    \usepackage{multirow}
    \usepackage{pbox}
    \usepackage{pifont}
    \usepackage{wrapfig}
    \usepackage[T1]{fontenc}
 
    
    % Header
    \pagestyle{fancy}
    \fancyhead{}
    \fancyfoot{}
    \fancyhead[L]{Formelsammlung Statistik SoSe 2018}
    \fancyhead[R]{Seite $\thepage$ von $\pageref{LastPage}$}
    \fancyheadoffset{0cm}
    
    % Document
    \setlength{\columnseprule}{0.5pt}
    \setlength{\topskip}{10pt} 
    \setlist{nosep}
    
    \titleformat*{\section}{\normalsize\bfseries}
    \titleformat*{\subsection}{\small\bfseries}
    \titleformat*{\subsubsection}{\small\bfseries}
    \titleformat*{\paragraph}{\bfseries}
    \titleformat*{\subparagraph}{\bfseries}
    
    \titlespacing*{\section}
    {0pt}{4pt}{0pt}
    \titlespacing*{\subsection}
    {0pt}{4pt}{0pt}
    \titlespacing*{\subsubsection}
    {0pt}{4pt}{0pt}
    \titlespacing*{\paragraph}
    {0pt}{4pt}{8pt}
    
    \newcolumntype{P}[1]{>{\centering\arraybackslash}p{#1}}
    \newcolumntype{M}[1]{>{\centering\arraybackslash}m{#1}}
    
    \makeatletter 
    \newcommand{\xRightarrow}[2][]{\ext@arrow 0359\Rightarrowfill@{#1}{#2}} 
    \makeatother 

    % Building blocks
    \newcommand{\heading}[1]{\noindent\section*{\framebox[\columnwidth][l]{#1}}}
    \newcommand{\subheading}[1]{\noindent\subsection*{\framebox[\columnwidth][l]{#1}}}
    \newcommand{\subsubheading}[1]{\noindent\framebox[\columnwidth][l]{#1}}
    \newcommand{\ccontent}[1]{\parbox{\columnwidth}{\centering{#1}}}
    
	
    % Content 
\begin{document}
\begin{multicols*}{4}

	\heading{Beschreibende Statistik}
	\subheading{Lageparameter}

	\subsubheading{Arithmetisches Mittel}
	\ccontent{$\overline{x}=\frac{1}{n}\sum \limits_{i=1}^n x_i=\frac{1}{n}(x_1 + \dots + a_n)$}
	\ccontent{Das Arithmetische Mittel $\overline{x}$ minimiert die Funktion}
	\ccontent{ $g(t) = \sum \limits_{i=1}^{n} (x_i - t)^2$}

	\subsubheading{Geometrisches Mittel}
	\ccontent{$\overline{x}_{geom} = \sqrt[n]{x_1 \cdot x_2 \cdot \dots \cdot x_n}$}

	\subsubheading{Median}
	\ccontent{$\tilde{x} = \begin{cases} x_\frac{n+1}{2} \\ \frac{1}{2} \cdot (x_\frac{n}{2} + x_{\frac{n}{2}+1}) \end{cases}$ $\begin{array}{l} , ungerade \\ , gerade \end{array}$}
	Der Median $\tilde{x}$ minimiert die Funktion
	\ccontent{$g(t) = \sum \limits_{i=1}^n \vert x_i - t \vert$}

	\subheading{Streungsmaße}
	\subsubheading{(empirische) Varianz}
	\ccontent{$var = \sigma^2 = {s_n}^2 = \frac{1}{n-1} \sum \limits_{i=1}^n(x_i-\overline{x})^2$}
	\ccontent{alternativ}
	\ccontent{$var = \sigma^2 = \frac{n}{n-1} \cdot (\overline{x^2} - \overline{x}^2)$}

	\subsubheading{Standardabweichung}
	\ccontent{$\sigma = s_n = \sqrt{\sigma^2}$}
	\ccontent{$\sigma = s_n = \sqrt{{s_n}^2}$}

	\subsubheading{mittlere absolute Abweichung}
	\ccontent{$\frac{1}{n} \sum \limits_{i=1}^n \vert x_i - \tilde{x} \vert$ für Median}
	\ccontent{$\frac{1}{n} \sum \limits_{i=1}^n \vert x_i - \overline{x} \vert$ für arithmetisches Mittel}

	\subheading{Kovarianz und Korrelationskoeffizient}
	\subsubheading{Kovarianz}
	\ccontent{$cov(x,y) = S_{xy} = \frac{1}{n-1}\sum \limits_{i=1}^n(x_i-\overline{x})\cdot(y_i-\overline{y})$}
	\ccontent{alternativ}
	\ccontent{$cov(x,y) = S_{xy} = \frac{1}{n-1}\sum \limits_{i=1}^{n} (x_i \cdot y_i - n \cdot \overline{x} \cdot \overline{y})$}

	\subsubheading{Korrelationskoeffizent}
	\ccontent{$r_{xy} = \frac{S_{xy}}{S_x \cdot S_y}$}
	Der Korrelationskoeffizent liegt immer zwischen $-1 \leq r \leq +1$. 
	Je näher $r_{xy}$ bei $-1$ (negative Korellation/Steigung), oder $+1$ (positive Steigung/Korrelation) 
	liegt, desto genauer schmiegen sich die Messwerte an eine Gerade an. Bei $r_{xy}$ nahe 0 gibt es keinen 
	\textit{linearen} Zusammenhang zwischen den Merkmalen.

	\subheading{Regressionsrechnung}
	\subsubheading{Regressionsgerade}
	\ccontent{ Variante 1}
	\ccontent{$y = \overline{y} + \frac{S_{xy}}{{\sigma_x}^2} \cdot (x-\overline{x})$}
	\ccontent{ Variante 2}
	\ccontent{$y = b + a \cdot x$}
	\ccontent{$a = \frac{S_{xy}}{{\sigma_x}^2}$ und $b = \overline{y} - a \cdot \overline{x}$}

	\subsubheading{Kleinste quadratische Abweichung}
	\ccontent{ Die Parameter $a, b, c, ...$ werden so gewählt, dass}
	\ccontent{ $Q(a,b,c,...) = \sum \limits_{i=1}^n (f_{a,b,c,...}(x_i) - y_i)^2$}
	\ccontent{ minimal ist}
	\ccontent{ $f_{a,b,c...}(x_i)$ ist die Funktion dessen Parameter gesucht werden}
	\ccontent{ Nullsetzen der partiellen Ableitungen:}
	\ccontent{ $\frac{\partial}{\partial a} Q(a,b) = 0 $}
	\ccontent{ $\frac{\partial}{\partial b} Q(a,b) = 0 $}
	\ccontent{ Über die Ableitungen lassen sich die Parameter finden welche die vorgegebene Funktion am besten annähern}
	
	\subsubheading{ Vergleich ermittelter Kurven}
	\ccontent{ Um Kurven zu vergleichen, einfach die ermittelten Parameter in die $Q(a,b,c,...)$ Funktion eingeben und Wert berechnen. Je kleiner der Wert desto besser passt die Kurve}

	\heading{Wahrscheinlichkeitstheorie}
	\subheading{Wahrscheinlichkeitsräume}
	\subsubheading{Der Wahrscheinlichkeitsbegriff}
	\ccontent{$Ergebnismenge = \Omega$}
	\ccontent{ Beispiel Würfel $\Omega = \{ 1, 2, 3, 4, 5, 6 \}$}
	\ccontent{ Ein Ereignis ist eine Teilmenge der Ergebnismenge}
	\ccontent{$\varnothing \, \subseteq \, \Omega \, \widehat{=}$ unmögliches Ereignis}
	\ccontent{$\Omega \, \subseteq \, \Omega \, \widehat{=}$ sicheres Ereignis}
	\ccontent{$A = \{1, 2, 3 \}$ Ereignis}
	\ccontent{$\overline{A} = \{4, 5, 6 \}$ Gegenereignis}

	\subsubheading{Elementarereignis}
	\ccontent{ einelementige Teilmenge von $\Omega$}
	\ccontent{ Ereignis, eine 3 werfen}
	\ccontent{ $B = \{ 3 \}$}
	\ccontent{ $P(\{3\}) = \frac{1}{6}$}

	\subsubheading{Laplace-Versuch}
	\ccontent{ Jedes Elementarereignis ist gleich wahrscheinlich}
	\ccontent{ $P(\{\omega_i\}) = \frac{1}{\vert \Omega \vert}$}
	\ccontent{ $P(A) = \frac{\vert A \vert}{\vert \Omega \vert} = \frac{3}{6} = \frac{1}{2}$}

	\subheading{Bedingte Wahrscheinlichkeit}
	\subsubheading{Bedingte Wahrscheinlichkeit}
	\ccontent{ Wahrscheinlichkeit für A unter der Bedingung B}
	\ccontent{ $P(A \vert B) = \frac{P(A \cap B)}{P(B)}$}
	\ccontent{ $P(\overline{A} \vert B) = 1 - P(A \vert B)$}

	\subsubheading{Formel von Bayes}
	\ccontent{ $P(A \vert B) = \frac{P(B \vert A) \cdot P(A)}{P(B)}$}

	\subsubheading{Satz der totalen Wahrscheinlichkeit}
	\ccontent{ $P(A) = \sum \limits_{i=1}^n (P(A \vert B_i) \cdot P(B_i)$}
	
	\subsubheading{Viel Felder Tafel}
	\def\arraystretch{1.5}% 
	\begin{tabularx}{\columnwidth}{l|X|X|r}
					&$A$					&$\overline{A}$					&$\sum$\\
		\cline{1-4}
		$B$			&$P(A \cap B)$			&$P(\overline{A} \cap B)$		&$P(B)$\\
		\cline{1-4}
		$\overline{B}$	&$P(A \cap \overline{B})$	&$P(\overline{A} \cap \overline{B})$	&$P(\overline{B})$\\
		\cline{1-4}
		$\sum$		&$P(A)$					&$P(\overline{A})$				&$1$\\
	\end{tabularx}
	\ccontent{ Die Ränder sind immer die Summen der zugehörigen Zeilen oder Spalten}

	\subsubheading{Allgemeine Regeln}
	\ccontent{ $P(A \cap B) = P(A \vert B) \cdot P(B) = P(B \vert A) \cdot P(A)$}
	\ccontent{ $P(A \cup B) = P(A) + P(B) - P(A \cap B)$}
	\ccontent{ $P(\overline{A}) = 1 - P(A)$}
	\ccontent{ $P(\overline{A \cup B}) = P(\overline{A} \cap \overline{B})$}
	\ccontent{ $P(\overline{A \cap B}) = P(\overline{A} \cup \overline{B})$}
	\ccontent{ Wenn A und B unabhängig, dann gilt}
	\ccontent{ $P(A \cap B) = P(A) \cdot P(B)$}
	\ccontent{ $P(A \vert B) = P(A)$}

	\subheading{Zufallsvariablen}
	\ccontent{ Eine Zufallsvariable ist eine Zuordnungsvorschrift die jedem  möglichen Ergebnis eines Zufallsexperiments eine Größe zuordnet}
	\ccontent{ $X = k \, \widehat{=} \, \{ \omega \in \Omega \vert X( \omega = k \}$}
	\ccontent{ $X = 3 \, \widehat{=} \, \{ \omega \in \Omega \vert X( \omega = 3 \}$}
	\ccontent{ $X \leq k \, \widehat{=} \, \{ \omega \in \Omega \vert X( \omega \leq k \}$}

	\subheading{Diskrete Verteilungen}
	\subsubheading{Binomialverteilung}
	\ccontent{ Mit zurücklegen, Wahrscheinlichkeit für jedes Ereignis gleich }
	\ccontent{ $X \sim B(n,p)$}
	\ccontent{ $n =:$ Stichprobenumfang}
	\ccontent{ $p =:$ Wahrscheinlichkeit}
	\ccontent{ ($p$ muss bei Binomialverteilung fest bleiben)}
	\ccontent{ $P(X=k) = \binom{n}{k} \cdot p^k \cdot (1-p)^{n-k}$}
	\ccontent{ $P(X \leq k) = \sum \limits_{i=0}^k\binom{n}{i} \cdot p^i \cdot (1-p)^{n-i}$}
	\ccontent{ $P(X > k) = 1 - P(X \leq k)$}
	\ccontent{ Eingabe Taschenrechner}
	\ccontent{ $\binom{n}{k} \, \widehat{=} \, n \, \vert nCr \vert \, k$}
	\ccontent{ $Mode \rightarrow 4 \rightarrow \downarrow \rightarrow 1  \rightarrow 2 \rightarrow k \rightarrow n \rightarrow p$ }
	
	\subsubheading{Binomialverteilung approximieren}
	\ccontent{ Die \textbf{Binomialverteilung} kann mit der \textbf{Poisson} Verteilung approximiert werden, dann gilt}
	\ccontent{ $\lambda = n \cdot p$}
	\ccontent{ Die \textbf{Binomialverteilung} kann auch mit der \textbf{Normalverteilung} approximiert werden, wobei gilt $n \cdot p = \mu$ und $n \cdot p \cdot (1-p) = \sigma^2$, \textbf{bedingung ist}}
	\ccontent{ $X \sim B(n, p) \approx N(n \cdot p, n \cdot p \cdot (1-p))$}
	\ccontent{ falls gilt}
	\ccontent{ $n \cdot p \cdot (1-p) > 9$}
	\ccontent{Bei der approximation mit der Normalverteilung kann man eine \textbf{Stetigkeitskorrektur} verwenden um ein besseres Ergebnis zu erhalten}
	\ccontent{$P(X \leq k) \approx F_N(R + 0,5)$}
	\ccontent{$P(X < k) \approx F_N(R - 0,5)$}
	\ccontent{$P(a \leq X \leq b) \approx F_N(b + 0,5) - F_N(a-0,5)$}
	\ccontent{\textbf{Zusammengefasst}}
	\ccontent{Falls $np$ und $n(1-p)$ groß genug sind:}
	\ccontent{$\mu = n\cdot p$ und $\sigma^2 = n\cdot p \cdot (1-p)$}
	\ccontent{$F_B(x) \approx F_N(x+0.5) = \Phi(\frac{x+0.5-np}{\sqrt{np(1-p)}})$}

	\subsubheading{Hypergeometrische Verteilung}
	\ccontent{ Ohne zurücklegen, Wahrscheinlichkeit ändert sich nach jedem Ereignis}
	\ccontent{ $X \sim H(N, M, n)$}
	\ccontent{ $n =:$ Stichprobenumfang}
	\ccontent{ $N =:$ Gesamtzahl}
	\ccontent{ $M =:$ Anzahl der Elemente mit der Eigenschaft}
	\ccontent{ $P(X = k) = \frac{\binom{M}{k} \cdot \binom{N-M}{n-k}}{\binom{N}{n}} $}
	\ccontent{ $P(X \leq k) = \sum \limits_{i=0}^k \frac{\binom{M}{i} \cdot \binom{N-M}{n-i}}{\binom{N}{n}} $}
	\ccontent{ $P(X > k) = 1 - P(X \leq k)$}
	
	\subheading{Hypergeometrische Vert. approximieren}
	\ccontent{Die hypergeometrische Verteilung kann mit der \textbf{Binomialverteilung} approximiert werden. Dabei muss folgende Bedingung gelten}
	\ccontent{ $\frac{n}{N} < 0,05$}

	\subsubheading{Poisson Verteilung}
	\ccontent{ Schlüsselwörter sind \textbf{Ereignisse pro Zeiteinheit}, zum Beispiel Anrufe innerhalb bestimmter Zeitspanne}
	\ccontent{ $X \sim Pois(\lambda)$}
	\ccontent{ $P(X = k) = \pi_\lambda(k) = \frac{\lambda^k}{k!} \cdot e^{-\lambda}$}
	\ccontent{ Eingabe Taschenrechner }
	\ccontent{ $Mode \rightarrow 4 \rightarrow \downarrow \rightarrow 3 \rightarrow 2 \rightarrow k \rightarrow \lambda$ }
	\subsubheading{Nährung an Normalverteilung}
	\ccontent{Wenn $\lambda$ groß genug ist kann die Verteilungsfunktion $F_P(x)$ der Poissonverteilung durch die Verteilungsfunktion der Normalverteilung $F_N(x)$ mit den Parametern}
	\ccontent{$\mu = \lambda$ und $\sigma^2 = \lambda$}
	\ccontent{genähert werden:}
	\ccontent{$F_P(x) \approx F_N(x+0.5) = \Phi(\frac{x+0.5-\lambda}{sqrt{\lambda}})$}

	\subsubheading{Geometrische Verteilung}
	\ccontent{ $X \sim Geom(n, p)$}
	\ccontent{ $P(X = n) = (1-p)^{n-1} \cdot p$}
	\ccontent{ Beispiel: Ein Würfel wird so lange gewürfelt bis eine 6 Auftritt. Die Zufallsvariable X ist gleich Anzahl der Würfe}

	\subheading{Stetige Verteilungen}

	\subsubheading{Dichtefunktion}
	\ccontent{Die Dichtefunktion ist ein Hilfsmittel zur \textbf{Beschreibung einer stetigen Wahrscheinlichkeitsverteilung}}
	\ccontent{Bedingungen der Dichtefunkion}
	\ccontent{ $f(x) \geq 0$}
	\ccontent{ $\int_{-\infty}^{\infty} f(x)dx = 1$}
	\ccontent{Die Dichtefunktion muss \textbf{nicht} stetig sein}
	\ccontent{Die Dichtefunktion ist die Ableitung der Verteilungsfunktion $F(x)$}
	
	\subsubheading{Verteilungsfunktion}
	\ccontent{Eine Verteilungsfunktion ist eine Funktion $F$, die jedem $x$ einer Zufallsvariable $X$ genau eine Wahrscheinlichkeit $P(X \leq x)$ zuordnet}
	\ccontent{ $F(x) \to P(X \leq x)$}
	\ccontent{Bedingungen der Verteilungsfunktion}
	\ccontent{Die Verteilungsfunktion \textbf{muss} stetig sein}
	\ccontent{Die Verteilungsfunktion \textbf{muss} monoton steigend sein}
	\ccontent{ $\lim \limits_{x \to \infty} F(x) = 1$}
	\ccontent{ $\lim \limits_{x \to -\infty} F(x) = 0$}
	
	\subsubheading{Normalverteilung}
	\ccontent{ $X \sim N(\mu, \sigma^2)$}
	\ccontent{ Ist $X \sim N(0, 1)$ dann heißt sie Standardnormalverteilt}
	\ccontent{ Jede Normalverteilung kann standardisiert werden, das heißt die Mitte der Kurve wird auf den Nullpunkt gesetzt}
	\ccontent{ Wenn $X \sim N(\mu, \sigma^2)$ verteilt ist}
	\ccontent{ dann ist die standardisierte Zufallsvariable $Z = \frac{x-\mu}{\sigma} \sim N(0,1)$ standardnormalverteilt}
	\ccontent{ Ist die Zufallsvariable standardverteilt kann die Wahrscheinlichkeit aus der Tabelle abgelesen werden}
	\subsubheading{Regeln für den \textit{Phi}-Wert}
	\ccontent{ $P(X \leq k) = \Phi(k)$}
	\ccontent{ $P(X \leq -k) = 1 - \Phi(+k)$}
	\ccontent{ $P(X = k) = 0$ (``Integral ohne Breite!'')}
	\ccontent{ allgemein folgt daraus, wenn}
	\ccontent{ $X \sim N (\mu, \sigma^2)$}
	\ccontent{dann gilt}
	\ccontent{ $P(X \leq k) = \Phi(\frac{k-\mu}{\sigma})$}
	\ccontent{ $P(a \leq X \leq b) = \Phi(\frac{b-\mu}{\sigma}) - \Phi(\frac{a-\mu}{\sigma})$}

	\subsubheading{Additionssatz der Normalverteilung}
	\ccontent{Seien $X$ und $Y$ unabhängig und Normalverteilt, dann gilt: }
	\ccontent{ $X + Y = N(\mu_X + \mu_Y; \sigma_X^2 + \sigma_Y^2) $ }
	\ccontent{Ihre Summe ist ebenfalls Normalverteilt! }

	\subsubheading{Quantile der Normalverteilung}
	\ccontent{ Tabelliert ist das $\beta-$Quantil $z_\beta$ der Normalverteilung $N(0,1)$}
	\ccontent{ $P(X \leq z_\beta) = \beta$}
	\ccontent{ $z_{1-\beta} = -z_\beta$}
	\ccontent{ Beispiel}
	\ccontent{ $\beta = 0.9 => z_\beta = 1.28155$}
	
	\subsubheading{Exponentialverteilung}
	\ccontent{ Eine exponentialverteilte Zufallsvariable T hat die Dichte}
	\ccontent{ $f(t) = \begin{cases} \lambda \cdot e^{- \lambda \cdot t} \\ 0 \end{cases}$ $\begin{array}{l} , t \geq 0 \\ , t < 0 \end{array}$}
	\ccontent{ und daraus eribt sich die Verteilungsfunktion}
	\ccontent{ $ F(x) = P(T \leq x) = $}
	\ccontent{ $ = \int_{-\infty}^{x} f(t)dt = \begin{cases} 1 - e^{- \lambda \cdot x}\\ 0\end{cases}$ $\begin{array}{l} , x \geq 0 \\ , x < 0\end{array}$}
	\ccontent{ Die Exponentialverteilung ist Gedächtnislos}
	
	\subsubheading{Gleichverteilung(Rechteckverteilung)}
	\ccontent{$f(t) = \begin{cases} \frac{1}{b-a} \\ 0 \end{cases}$ $\begin{array}{l}, t \in [a, b] \\ , sonst \end{array}$}
	\ccontent{$F(t) = \begin{cases} 0 \\ \frac{t-a}{b-a} \\ 1 \end{cases}$ $\begin{array}{l} , t < a \\, t \in [a, b] \\, t > b \end{array}$}
	
	\subheading{Erwartungswert und Varianz}
	\subsubheading{\textbf{Erwartungswert}}	
	\ccontent{ Erwartungswert und Mittelwert sind prinzipiell gleichwertig, der Erwartungswert entspricht der theoretischen Erwartung, der Mittelwert entspricht den tatsächlichen Werten}

	\subsubheading{Zufallsvariable mit diskreter Verteilung}
	\ccontent{ $\mu = E(X) = \sum \limits_{i=0}^n (x_i \cdot p_i)$}

	\subsubheading{Zufallsvariable mit Dichtefunktion $f$}
	\ccontent{ $\mu = E(X) = \int_{-\infty}^{\infty} x \cdot f(x) dx $}
	\ccontent{ $E(X^2) = \int_{-\infty}^{\infty} x^2 \cdot f(x) dx$}
	
	\subsubheading{Exponentialverteilung mit Zufallsvariable T}
	\ccontent{ $E(T) = \sigma_T = \frac{1}{\lambda}$}

	\subsubheading{Für Binomialverteilung}
	\ccontent{ $\mu = E(X) = n \cdot p$}

	\subsubheading{Für geometrische Verteilung}
	\ccontent{ $\mu = E(X) = \frac{1}{p}$}

	\subsubheading{Für Poissonverteilung}
	\ccontent{ $\mu = E(X) = \lambda$}

	\subsubheading{Für Hypergeometrischeverteilung}
	\ccontent{ $E(S_n) = E(X_1 + ... + X_n) = n \cdot E(X_1) = n \cdot \frac{M}{N}$}
	
	\subsubheading{Für Gleichverteilung(Rechteckverteilung)}
	\ccontent{ $E(T_i) = \frac{a+b}{2}$}

	\subsubheading{Allgemeine Regeln für den Erwartungswert}
	\ccontent{ $a, b \in \mathbb{R}$}
	\ccontent{ $E(aX + b) = a \cdot E(X) + b$}
	\ccontent{ $E(X + Y) = E(X) + E(Y)$}
	\ccontent{ $E(aX + bY) = a \cdot E(X) + b \cdot E(Y)$}

	\subsubheading{\textbf{Varianz}}

	\subsubheading{Zufallsvariable mit diskreter Verteilung}
	\ccontent{ $\sigma^2 = Var(X) = \sum (x_i - \mu)^2 \cdot p_i$}

	\subsubheading{Zufallsvariable mit Dichtefunktion $f$}
	\ccontent{ $Var(X) = E(X^2) - (E(X))^2$}
	\begin{minipage}[b]{\linewidth}
	\subsubheading{Varianz aus Erwartungswert berechnen}
	\ccontent{ $\sigma^2=E(X^2)-\mu^2$ }
	\end{minipage}

	\subsubheading{Exponentialverteilung mit Zufallsvariable T}
	\ccontent{ $Var(T) = \frac{1}{\lambda^2}$}

	\subsubheading{Für Binomialverteilung}
	\ccontent{ $\sigma^2 = n \cdot p \cdot (1-p)$}

	\subsubheading{Für geometrische Verteilung}
	\ccontent{ $\sigma^2 = \frac{1}{p^2} - \frac{1}{p}$}

	\subsubheading{Für Poissonverteilung}
	\ccontent{ $\sigma^2 = Var(X) = E(X^2) - E(X)^2 = \lambda$}

	\subsubheading{Für Hypergeometrischeverteilung}
	\ccontent{ $Var(S_n) = n \cdot \frac{M}{N} \cdot (1 - \frac{M}{N}) \cdot \frac{N-n}{N-1}$}
	
	\subsubheading{Für Gleichverteilung(Rechteckverteilung)}
	\ccontent{$ Var(T_i) = \frac{(b-a)^2}{12}$}
	\begin{minipage}[b]{\linewidth}
	\subsubheading{Allgemeine Regeln für Varianz}
	\ccontent{ $Var(X+b) = Var(X)$}
	\ccontent{ $Var(aX+b) = a^2 \cdot Var(X)$}
	\ccontent{ $Var(X + Y) = Var(X) + Var(Y) + 2 \cdot cov(X,Y)$}
	\ccontent{ wobei gilt: }
	\ccontent{ $Cov(X,Y) = E((X-\mu_X)(Y-\mu_Y)) = E(X \cdot Y)-\mu_X\mu_Y$ }
	\ccontent{ bei unabhängigen Zufallsvariablen $X$ und $Y$ ist $Cov(X,Y) = 0$ siehe unten.}
	\end{minipage}

	\subsubheading{\textbf{Unabhängige Zufallsvariablen}}
	\subsubheading{Allgemeine Regeln}
	\ccontent{ $Var(X + const) = Var(X)$}
	\ccontent{ $Var(X + Y) = Var(X) + Var(Y)$}
	\ccontent{ $E(X \cdot Y) = E(X) \cdot E(Y)$}
	
	\subheading{Wichtige Sätze der Stochastik}
	\subsubheading{Zentraler Grenzwertsatz}
	\ccontent{ n groß (Anzahl der Zufallsvariablen) $n \geq 30$}
	\ccontent{ $X_i$ unabhängig und identisch verteilt}
	\ccontent{ $\widehat{=}$ haben die gleiche Verteilung}
	\ccontent{ $E(X_i) = \mu$}
	\ccontent{ $Var(X_i) = \sigma^2$}
	\ccontent{ $\sum X_i \sim N(n \cdot \mu, n \cdot \sigma^2)$}
	\ccontent{ $\overline{X}_n = \frac{X_1+ ... + X_n}{n} = \overline{x} \sim N(\mu, \frac{\sigma^2}{n})$}
	\ccontent{ Manche Verteilungen verhalten sich in der Summe anders, zum Beispiel die Rechteckverteilung ist nicht mehr R-Verteilt. Dann wird der Zentrale Grenzwertsatz verwendet}
	\subsubheading{ ZGS - Definition }
	\ccontent{ Seien $X_1,...,X_n$ unabhängige und identisch verteilte Zufallsvariablen (nicht zwangsläufig Normalverteilt) mit Erwarungswert $\mu$ und Varianz $\sigma^2$. Ihre Summe sei $S = X_1 + ... + X_n$ mit Erwarungswert $n\mu$ und Varianz $n\sigma^2$. Es gilt für die zugehörige Zufallsvariable }
	\ccontent{$Z = \frac{S-n\mu}{\sqrt{n}\sigma} = \frac{\overline{X}-\mu}{\frac{\sigma}{\sqrt{n}}}$ gilt $\lim\limits_{n \rightarrow \infty}{P(Z\leq z)} = \Phi(z)$} 
	\heading{Induktive Statistik - Schätztheorie}
	\subheading{Schätzfunktionen}
	\begin{minipage}{\columnwidth}
	\subsubheading{Maximum-Likelihood-Schätzer}
	\ccontent{Vorbereitung: Falls $F(x)$ gegen leite ab um $f(x)$ zu erhalten ($f(x_i)$ muss eine \textbf{Dichtefunktion} sein)}
	\ccontent{1. Stelle auf: $L(x_1, \dots , x_n, \alpha) = \prod \limits_{i=1}^n f(x_i)$ \\
	\textit{Tipp}: Wenn möglich alles was nicht von $x_i$ abhängt aus der Summenfunktion ziehen. \\
	2. Wende an: ln($L(x_1, \dots , x_n, \alpha)$)\\
	3. Die Funktion nach dem Parameter $\alpha$ ableiten und Nullsetzen: $ \frac{\partial \ln{L(x_1, \dots , x_n, \alpha)}}{\partial \alpha} = 0$ \\
	4. Nach $\alpha$ auflösen, Das Ergebnis ist der Maximum-Likelihood-Schätzer
	}
	\end{minipage}
	
	\subheading{Konfidenzintervalle}
	\subsubheading{Intervall für $E(X)$ einer Normalverteilung}
	\ccontent{Ist $X \sim N(\mu, \sigma^2)$ verteilt, dann ist}
	\ccontent{$Z = \frac{\overline{X} - \mu}{\sigma / \sqrt{n}} \sim N(0,1)$}
	\subsubheading{Bei \textbf{bekannter} Standardabweichung $\sigma$}
	\ccontent{$\left[\overline{x} - z_{1-(\alpha / 2)} \cdot \frac{\sigma}{\sqrt{n}}, \overline{x} + z_{1-(\alpha / 2)} \cdot \frac{\sigma}{\sqrt{n}}\right]$}
	\ccontent{$\overline{X} = $ arithmetisches Mittel, bzw erwartungsstreuer Schätzer bei $n$ unabhängigen Stichproben}
	\ccontent{$\alpha = $ Signifikanzwahrscheinlichkeit (Irrtumswahrscheinlichkeit)}
	\ccontent{$1-\alpha = $ Vertrauensniveau}
	\ccontent{Ist $\alpha$ gegegeben, berechne das Quantil $z_{1-(\alpha / 2)}$}
	
	\subsubheading{Bei \textbf{unbekannter} Standardabweichung $\sigma$}
	\ccontent{$\left[\overline{x} - t_{n-1;1-(\alpha / 2)} \cdot \frac{s}{\sqrt{n}}, \overline{x} + t_{n-1;1-(\alpha / 2)} \cdot \frac{s}{\sqrt{n}}\right]$}
	\ccontent{Anstatt $\sigma^2$ wird der erwartungstreue Schätzer $s$ verwendet}
	\ccontent{ $s^2 = \frac{1}{n-1} \sum \limits_{i=1}^n(X_i - \overline{X})^2$}
	\ccontent{ beziehungsweise }
	\ccontent{ $s = \sqrt{\frac{1}{n-1} \sum \limits_{i=1}^n(X_i - \overline{X})^2}$}

	\heading{Induktive Statistik - Hypothesentest}
	\subheading{Tests für Lageparameter}
	\ccontent{Wähle Fall a) wenn Verteilung statt absoluter Wert gegeben (z.b. "jeder n-te...") und verfolge Stragie (3).}
	\ccontent{Wähle Fall b) wenn $\overline{X} \leq \mu_0$}
	\ccontent{Wähle Fall c) wenn $\overline{X} \geq \mu_0$}

	\subsubheading{\textbf{(1)}Gauß-Test (Wählen wenn $\sigma^2$ bekannt)}
	\ccontent{Ist ein Test für den Erwartungswert einer \textit{Normalverteilung} bei bekannter Standardabweichung $\sigma$}
	\ccontent{Wähle eine mögliche Hypothesenkombination:}
	\ccontent{a) $H_0:\mu = \mu_0$}
	\ccontent{b) $H_0:\mu \leq \mu_0$}
	\ccontent{c) $H_0:\mu \geq \mu_0$}
	\ccontent{Wähle ein Signifikanzniveau $\alpha$ (z.B.: 0.05)}
	\ccontent{Ziehe eine Stichprobe vom Umfang $n$, berechne $\overline{x}$ und den zugehörigen standardisierten Prüfwert:}
	\ccontent{$z=\sqrt{n}\cdot\frac{\overline{x}-\mu_0}{\sigma}$}
	\ccontent{Bestimme das entsprechende Quantil der Standartnormalverteilung:}
	\ccontent{a) $z_{1-\frac{\alpha}{2}}$ bzw. b) $z_{1-\alpha}$ c) $-z_{1-\alpha}$}
	\ccontent{$H_0$ ist zu verwerfen, falls}
	\ccontent{a) $|z| > z_{1-\frac{\alpha}{2}}$}
	\ccontent{b) $z > z_{1-\alpha}$}
	\ccontent{c) $z < -z_{1-\alpha}$}

	\subsubheading{Gauß-Test Tabellenform}
	\def\arraystretch{1.5}% 
	\begin{tabularx}{\columnwidth}{l|c|c}
		 & $H_0$ & Ablehnungsbereich \\
		 \cline{1-3}
		 & $ \mu = \mu_0$ & $\vert T \vert > Z_{1-\frac{\alpha}{2}}$ \\
		 \cline{1-3}
		 $\overline{X} \leq \mu_0$ & $\mu \leq \mu_0$ & $T > Z_{1-\alpha}$ \\
		 \cline{1-3}
		 $\overline{X} \geq \mu_0$ & $\mu \geq \mu_0$ & $T < -Z_{1-\alpha}$ \\
	\end{tabularx}

	\subsubheading{\textbf{(2)}$t$-Test (Wählen wenn $\sigma^2$ \textbf{nicht} bekannt)}
	\ccontent{Wähle eine mögliche Hypothesenkombination:}
	\ccontent{a) $H_0:\mu = \mu_0$}
	\ccontent{b) $H_0:\mu \leq \mu_0$}
	\ccontent{c) $H_0:\mu \geq \mu_0$}
	\ccontent{Wähle ein Signifikanzniveau $\alpha$ (z.B.: 0.05)}
	\ccontent{Ziehe eine Stichprobe vom Umfang $n$, berechne daraus $\overline{x}$ und $s$ sowie den zugehörigen Prüfwert:}
	\ccontent{$t=\sqrt{n}\cdot\frac{\overline{x}-\mu_0}{s}$}
	\ccontent{Bestimme das entsprechende Quantil der $t$-Verteilung, wobei der $n$-Wert der Tabelle $=(n-1)$ ist, also die Stückzahl $-1$ ist:}
	\ccontent{a) $t_{n-1;1-\frac{\alpha}{2}}$ bzw. b) $t_{n-1;1-\alpha}$ c) $-t_{n-1;1-\alpha}$}
	\ccontent{$H_0$ ist zu verwerfen, falls}
	\ccontent{a) $|t| > t_{n-1;1-\frac{\alpha}{2}}$}
	\ccontent{b) $t > t_{n-1;1-\alpha}$}
	\ccontent{c) $t < -t_{n-1;1-\alpha}$}
	\subsubheading{$t$-Test Tabellenform}
	\begin{tabularx}{\columnwidth}{l|c|c|c}
		& $H_0$ & \multicolumn{2}{c}{Ablehnungsbereich} \\
		\cline{1-4}
		& $ \mu = \mu_0$ & $\vert T \vert > t_{1-\frac{\alpha}{2}}$ &$(n-1)$\\
		\cline{1-4}
		$\overline{X} \leq \mu_0$ & $\mu \leq \mu_0$ & $T > t_{1-\alpha}$ &$(n-1)$\\
		\cline{1-4}
		$\overline{X} \geq \mu_0$ & $\mu \geq \mu_0$ & $T < -t_{1-\alpha}$ &$(n-1)$\\
	\end{tabularx}

	\subsubheading{\textbf{(3)}Wahrscheinlichkeit gegeben}
	\ccontent{
		\begin{equation}
			X_i=
			\begin{cases}
			  1, & \text{\textit{i}te Stichprobe hat die Eigeschaft}\ \\
			  0, & \text{sonst}\nonumber
			\end{cases}
		  \end{equation}
	}
	\ccontent{$X_i \sim B(1, p_0)$}
	\ccontent{$\Rightarrow$ ZGS $\Rightarrow \sum X_i \stackrel{a}{\sim} N(np_0|np_0(1-p_0)$}
	\ccontent{$T=\frac{X-np_0}{\sqrt{np_0(1-p_0)}}$}
	\ccontent{Verwerfe oder nicht nach Regeln in Variante (1)}
	
	\subheading{Tests für Streuungsmaße}
	
	\subsubheading{$\chi^2$ - Anpassungstest}
	\ccontent{ Der $\chi^2$-Anpassungstest überprüft ob eine unbekannte Wahrscheinlichkeitsverteilung einem bestimmten Verteilungsmodell folgt}
	\ccontent{$ T = \frac{1}{n}(\sum \limits_{i=1}^r \frac{{N_i}^2}{p_i}) - n$}
	\ccontent{ wobei $N_i$ absolute Häufigkeit in Kategorie $r$, $n$ ist der Stichprobenumfang. Alternativ}
	\ccontent{ $T = \sum \limits_{i=1}^r \frac{(N_i - np_i)^2}{n \cdot p_i}$}
	\ccontent{ Das Ergebnis $T$ mit $\chi^2_{r-1;1-\alpha}$ Wert aus der Tabelle vergleichen}
	\ccontent{$T < \chi^2 = $ Hypothese wird nicht verworfen}
	\textbf{Algorithmus} \\
	1. Hypothese aufstellen \\
	2. $n$ und $r$ festlegen ($n$ = Stichprobenumfang, $r$ = Anzahl der Klassen) \\
	3. Verteilung auf welche getestet werden soll bestimmten \\
	4. Alle $p_i$ berechenen (zu jeder Kategorie $r$ eines nach in 3. festgelegter Verteilung) \\
	5. T berechnen. \\
	6. $\chi^2$ mit T vergleichen.
	\heading{Allgemeine Matheregeln}
	\subheading{Potenzen und Logarithmen}
	
	\subsubheading{Potenzgesetze}
	\def\arraystretch{0}%
	\begin{tabularx}{\columnwidth}{|X|}
		\cline{1-1}
		\begin{center}$a^0 = 1$\end{center}													\\
		\cline{1-1}
		\begin{center}$a^1 = a$\end{center}															\\
		\cline{1-1}
		\begin{center}$a^m \cdot a^n = a^{m+n}$\end{center}											\\
		\cline{1-1}
		\begin{center}$(a^n)^m = a^{n \cdot m}$\end{center}											\\
		\cline{1-1}
		\begin{center}$a^n \cdot b^n = (a \cdot b)^n$\end{center}									\\
		\cline{1-1}
		\begin{center}$\frac{a^n}{a^m} = a^{n-m}$\end{center}										\\
		\cline{1-1}
		\begin{center}$\prod \limits_{i=1}^n a^{x_i} = a^{\sum \limits_{i=1}^n x_i}$\end{center}	\\
		\cline{1-1}
		\begin{center}$\sqrt[n]{k} = k^{\frac{1}{n}}$\end{center}	\\
		\cline{1-1}	
	\end{tabularx}
	
	\subsubheading{Logarithmusregeln}
	\def\arraystretch{0}%
	\begin{tabularx}{\columnwidth}{|X|}
		\cline{1-1}
		\begin{center}$x = \log_{a}{y} \Leftrightarrow y = a^x$\end{center}	\\
		\cline{1-1}
		\begin{center}$\log{1} = 0$\end{center}								\\
		\cline{1-1}
		\begin{center}$\log{x \cdot y} = \log{x} + \log{y}$\end{center}		\\
		\cline{1-1}
		\begin{center}$- \log{x} = \log{\frac{1}{x}}$\end{center}			\\
		\cline{1-1}
		\begin{center}$\log{\frac{x}{y}} = \log{x} - \log{y}$\end{center}	\\
		\cline{1-1}
		\begin{center}$\log{x^n} = n \cdot \log{x}$\end{center}				\\
		\cline{1-1}
		\begin{center}$\log_{a}{x} = \frac{\log{x}} {\log{a}}$\end{center}	\\
		\cline{1-1}
		\begin{center}$\log{(\prod \limits_{i=1}^n x_i)} = \sum \limits_{i=1}^n \log{x_i}$\end{center} \\
		\cline{1-1}
		\begin{center}$ln(e^x) = x$\end{center} \\
		\cline{1-1}
	\end{tabularx}
	
	\subheading{Ableitungen und Integrale}

	\subsubheading{Grundlegende Ableitungsregeln}
	\def\arraystretch{1.5}% 
	\begin{tabularx}{\columnwidth}{|X|X|}
		\cline{1-2}
		$f(x)$      & $f'(x)$                    \\
		\cline{1-2}
		$c = const$ & $0$                        \\
		\cline{1-2}
		$x^n$       & $n \cdot x^{n-1}$          \\
		\cline{1-2}
		$\sqrt{x}$  & $\frac{1}{2\sqrt{x}}$      \\
		\cline{1-2}
		$e^x$       & $e^x$                      \\
		\cline{1-2}
		$a^x$       & $\ln a \cdot a^x$          \\
		\cline{1-2}
		$\ln x$     & $\frac{1}{x}$              \\
		\cline{1-2}
		$\log_a x$  & $\frac{1}{\ln  a \cdot x}$ \\
		\cline{1-2}
		$\sin{x}$   & $\cos{x}$                  \\
		\cline{1-2}
		$\cos{x}$   & $-\sin{x}$                 \\
		\cline{1-2}
		$\tan{x}$   & $\frac{1}{\cos^2{x}}$      \\
		\cline{1-2}
		$\cot{x}$   & $\frac{1}{\sin^2{x}}$      \\
		\cline{1-2}
	\end{tabularx}

	\subsubheading{Verknüpfte Ableitungsregeln}
	\def\arraystretch{1.5}% 
	\begin{tabularx}{\columnwidth}{|l|X|}
		\cline{1-2}
		$f(x)$              & $f'(x)$                                                  \\
		\cline{1-2}
		$(f(x)+g(x))$       & $(f'(x)+g'(x))$                                          \\
		\cline{1-2}
		$(f(x) \cdot g(x))$ & $(f'(x) \cdot g(x)) + (f(x) \cdot g'(x))$                \\
		\cline{1-2}
		$\frac{f(x)}{g(x)}$ & $\frac{(f'(x) \cdot g(x)) - (f(x) \cdot g'(x))}{g(x)^2}$ \\
		\cline{1-2}
		$f(g(x))$          & $f'(g(x)) \cdot g'(x)$                                    \\
		\cline{1-2}
	\end{tabularx}

	\subsubheading{wichtige Stammfunktionen}
	\def\arraystretch{1.5}% 
	\begin{tabularx}{\columnwidth}{|X|X|}
		\cline{1-2}
		$f(x)$                  & $F(x)$                                  \\
		\cline{1-2}
		$x^n, n \neq 1$         & $\frac{1}{n+1} \cdot x^{n+1} + c$       \\
		\cline{1-2}
		$c$         			& $cx + c$						          \\
		\cline{1-2}
		$\frac{1}{x}, x \neq 0$ & $\ln \vert x \vert + c$                 \\
		\cline{1-2}
		$\sqrt{x}$              & $\frac{2}{3} \cdot x^{\frac{3}{2}} + c$ \\
		\cline{1-2}
		$e^x$                   & $e^x + c$                               \\
		\cline{1-2}
	\end{tabularx}
	
	\subsubheading{Bestimmte Integrale}
	\ccontent {$$\int_{a}^{b} f(x)dx = \left[ F(x) + C \right]_{a}^{b} = F(b) - F(a)$$}
	\ccontent {$\int_{a}^{b} a \cdot f(x)dx = a \cdot \int_{a}^{b} f(x)dx$}
	
	\subsubheading{Summen und Produkte}
	\def\arraystretch{0}%
	\begin{tabularx}{\columnwidth}{|X|}
	\cline{1-1}
	\begin{center}$\prod\limits_{i=1}^n a \cdot x_i = a^n \prod\limits_{i=1}^n x_i$\end{center}	\\
	\cline{1-1}
	\begin{center}$\sum \limits_{i=1}^n a \cdot x_i = a \cdot \sum \limits_{i=1}^n x_i$\end{center} \\
	\cline{1-1}	
	\end{tabularx}

	\subsubheading{Sonstiges}
	\def\arraystretch{0}%
	\begin{tabularx}{\columnwidth}{|X|}
	\cline{1-1}
	\begin{center}$\frac{b}{\alpha} = c \Leftrightarrow \alpha = \frac{b}{c}$\end{center}	\\
	\cline{1-1}
	\end{tabularx}
\end{multicols*}
\end{document}