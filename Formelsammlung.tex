\documentclass[10pt,a4paper,landscape]{article}
\usepackage[left=0.55cm,right=0.55cm,top=1.10cm,bottom=0.55cm,landscape,
headsep=2mm]{geometry}

\usepackage{lastpage}
\usepackage{fancyhdr}
\usepackage{multicol}
\usepackage[utf8]{inputenc}
\usepackage[ngerman]{babel}
\usepackage[T1]{fontenc}
\usepackage{listings}
\usepackage{enumitem}
\setitemize{leftmargin=15pt}  
\setenumerate{leftmargin=15pt}  
\usepackage{titlesec}
\usepackage{color,soul}
\usepackage{graphicx}
\usepackage{tabularx}
\usepackage{tikz}
\usetikzlibrary{automata,positioning}
\usepackage[babel,german=quotes]{csquotes}
\usepackage{arydshln}
\usepackage[fleqn]{amsmath}
\usepackage{setspace}
\usepackage{amssymb}
\usepackage{float}
\usepackage{booktabs}
\usepackage{multirow}
\usepackage{pbox}
\usepackage{pifont}
\usepackage{wrapfig}
\usepackage[T1]{fontenc}

% Header
\pagestyle{fancy}
\fancyhead{}
\fancyfoot{}
\fancyhead[L]{Formelsammlung Statistik SoSe 2018}
\fancyhead[R]{Seite $\thepage$ von $\pageref{LastPage}$}
\fancyheadoffset{0cm}

% Document
\setlength{\columnseprule}{0.5pt}
\setlength{\topskip}{10pt} 
\setlist{nosep}

\titleformat*{\section}{\normalsize\bfseries}
\titleformat*{\subsection}{\small\bfseries}
\titleformat*{\subsubsection}{\small\bfseries}
\titleformat*{\paragraph}{\bfseries}
\titleformat*{\subparagraph}{\bfseries}

\titlespacing*{\section}
{0pt}{4pt}{0pt}
\titlespacing*{\subsection}
{0pt}{4pt}{0pt}
\titlespacing*{\subsubsection}
{0pt}{4pt}{0pt}
\titlespacing*{\paragraph}
{0pt}{4pt}{8pt}

\newcolumntype{P}[1]{>{\centering\arraybackslash}p{#1}}
\newcolumntype{M}[1]{>{\centering\arraybackslash}m{#1}}

\makeatletter 
\newcommand{\xRightarrow}[2][]{\ext@arrow 0359\Rightarrowfill@{#1}{#2}} 
\makeatother 

% Content 
\begin{document}
\begin{multicols*}{4}

\section{\framebox[\columnwidth][l]{Beschreibende Statistik}}
\subsection{\framebox[\columnwidth][l]{Lageparameter}}

\framebox[\columnwidth][l]{Arithmetisches Mittel}
\parbox{\columnwidth}{\centering$\bar{x}=\frac{1}{n}\sum \limits_{i=1}^n x_i=\frac{1}{n}(x_1 + \dots + a_n)$}

\framebox[\columnwidth][l]{Geometrisches Mittel}
\parbox{\columnwidth}{\centering$\bar{x}_{geom} = \sqrt[n]{x_1 \cdot x_2 \cdot \dots \cdot x_n}$}

\framebox[\columnwidth][l]{Median}
\parbox{\columnwidth}{\centering\[\tilde{x}=\begin{cases}x_{\frac{n+1}{2}}\\\frac{1}{2}(x_{\frac{n}{2}}+x_{\frac{n}{2}+1})\end{cases}\]}
Der Median $\tilde{x}$ minimiert die Funktion
\parbox{\columnwidth}{\centering$\sum \limits_{i=1}^n|x_i- \bar{x}|$}

\subsection{\framebox[\columnwidth][l]{Streungsmaße}}

\framebox[\columnwidth][l]{(empirische) Varianz}
\parbox{\columnwidth}{\centering$var = \sigma^2 = {s_n}^2 = \frac{1}{n-1} \sum \limits_{i=1}^n(x_i-\tilde{x})^2$}

\framebox[\columnwidth][l]{Standardabweichung}
\parbox{\columnwidth}{\centering$\sigma = \sqrt{\sigma^2}$}
\parbox{\columnwidth}{\centering$\sigma = \sqrt{{s_n}^2}$}

\subsection{\framebox[\columnwidth][l]{Kovarianz und Korrelationskoeffizient}}

\framebox[\columnwidth][l]{Kovarianz}
\parbox{\columnwidth}{\centering$cov(x,y) = S_{xy} = \frac{1}{n-1}\sum \limits_{i=1}^1(x_i-\bar{x})\cdot(y_i-\bar{y})$}

\framebox[\columnwidth][l]{Korrelationskoeffizent}
\parbox{\columnwidth}{\centering$r_{xy} = \frac{S_{xy}}{S_x \cdot S_y}$}
Der Korrelationskoeffizent liegt immer zwischen $-1 \leq r \leq 1$

\subsection{\framebox[\columnwidth][l]{Regressionsrechnung}}

\framebox[\columnwidth][l]{Regressionsgerade}
\parbox{\columnwidth}{\centering Variante 1}
\parbox{\columnwidth}{\centering$y = \bar{y} + \frac{S_{xy}}{{\sigma_x}^2} \cdot (x-\bar{x})$}
\parbox{\columnwidth}{\centering Variante 2}
\parbox{\columnwidth}{\centering$y = b + a \cdot x$}
\parbox{\columnwidth}{\centering$a = \frac{S_{xy}}{{\sigma_x}^2}$ und $b = \bar{y} - a \cdot \bar{x}$}

\framebox[\columnwidth][l]{kleinste quadratischen Abweichung}

\section{\framebox[\columnwidth][l]{Wahrscheinlichkeitstheorie}}
\subsection{\framebox[\columnwidth][l]{Wahrscheinlichkeitsräume}}

\framebox[\columnwidth][l]{Der Wahrscheinlichkeitsbegriff}
\parbox{\columnwidth}{\centering$Ergebnismenge = \Omega$}
\parbox{\columnwidth}{\centering Beispiel Würfel $\Omega = \{ 1, 2, 3, 4, 5, 6 \}$}
\parbox{\columnwidth}{\centering Ein Ereignis ist eine Teilmenge der Ergebnismenge}
\parbox{\columnwidth}{\centering$\varnothing \, \subseteq \, \Omega \, \widehat{=}$ unmögliches Ereignis}
\parbox{\columnwidth}{\centering$\Omega \, \subseteq \, \Omega \, \widehat{=}$ sicheres Ereignis}
\parbox{\columnwidth}{\centering$A = \{1, 2, 3 \}$ Ereignis}
\parbox{\columnwidth}{\centering$\bar{A} = \{4, 5, 6 \}$ Gegenereignis}

\framebox[\columnwidth][l]{Elementarereignis}
\parbox{\columnwidth}{\centering einelementige Teilmenge von $\Omega$}
\parbox{\columnwidth}{\centering Ereignis, eine 3 werfen}
\parbox{\columnwidth}{\centering $B = \{ 3 \}$}
\parbox{\columnwidth}{\centering $P(\{3\}) = \frac{1}{6}$}

\framebox[\columnwidth][l]{Laplace-Versuch}
\parbox{\columnwidth}{\centering Jedes Elementarereignis ist gleich wahrscheinlich}
\parbox{\columnwidth}{\centering $P(\{\omega_i\}) = \frac{1}{\vert \Omega \vert}$}
\parbox{\columnwidth}{\centering $P(A) = \frac{\vert A \vert}{\vert \Omega \vert} = \frac{3}{6} = \frac{1}{2}$}

\subsection{\framebox[\columnwidth][l]{Bedingte Wahrscheinlichkeit}}

\framebox[\columnwidth][l]{Bedingte Wahrscheinlichkeit}
\parbox{\columnwidth}{\centering Wahrscheinlichkeit für A unter der Bedingung B}
\parbox{\columnwidth}{\centering $P(A \vert B) = \frac{P(A \cap B)}{P(B)}$}
\parbox{\columnwidth}{\centering $P(\bar{A} \vert B) = 1 - P(A \vert B)$}

\framebox[\columnwidth][l]{Formel von Bayes}
\parbox{\columnwidth}{\centering $P(A \vert B) = \frac{P(B \vert A) \cdot P(A)}{P(B)}$}

\framebox[\columnwidth][l]{Satz der totalen Wahrscheinlichkeit}
\parbox{\columnwidth}{\centering $P(A) = \sum \limits_{i}^n (P(A \vert B_i) \cdot P(B_i)$}

\framebox[\columnwidth][l]{Allgemeine Regeln}
\parbox{\columnwidth}{\centering $P(A \cap B) = P(A \vert B) \cdot P(B) = P(B \vert A) \cdot P(A)$}
\parbox{\columnwidth}{\centering $P(A \cup B) = P(A) + P(B) - P(A \cap B)$}
\parbox{\columnwidth}{\centering $P(\bar{A}) = 1 - P(A)$}
\parbox{\columnwidth}{\centering $P(\bar{A \cup B}) = P(\bar{A} \cap \bar{B})$}
\parbox{\columnwidth}{\centering $P(\bar{A \cap B}) = P(\bar{A} \cup \bar{B})$}
\parbox{\columnwidth}{\centering Wenn A und B unabhängig, dann gilt}
\parbox{\columnwidth}{\centering $P(A \cap B) = P(A) \cdot P(B)$}

\subsection{\framebox[\columnwidth][l]{Zufallsvariablen}}

\parbox{\columnwidth}{\centering Eine Zufallsvariable ist eine Zuordnungsvorschrift die jedem  möglichen Ergebnis eines Zufallsexperiments eine Größe zuordnet}
\parbox{\columnwidth}{\centering $X = k \, \widehat{=} \, \{ \omega \in \Omega \vert X( \omega = k \}$}
\parbox{\columnwidth}{\centering $X = 3 \, \widehat{=} \, \{ \omega \in \Omega \vert X( \omega = 3 \}$}
\parbox{\columnwidth}{\centering $X \leq k \, \widehat{=} \, \{ \omega \in \Omega \vert X( \omega \leq k \}$}

\subsection{\framebox[\columnwidth][l]{Diskrete Verteilungen}}

\framebox[\columnwidth][l]{Binomialverteilung}
\parbox{\columnwidth}{\centering Mit zurücklegen, Wahrscheinlichkeit für jedes Ereignis gleich }
\parbox{\columnwidth}{\centering $X \sim B(n,p)$}
\parbox{\columnwidth}{\centering $n =:$ Stichprobenumfang}
\parbox{\columnwidth}{\centering $p =:$ Wahrscheinlichkeit}
\parbox{\columnwidth}{\centering ($p$ muss bei Binomialverteilung fest bleiben)}
\parbox{\columnwidth}{\centering $P(X=k) = \binom{n}{k} \cdot p^k \cdot (1-p)^{n-k}$}
\parbox{\columnwidth}{\centering $P(X \leq k) = \sum \limits_{i=0}^k\binom{n}{i} \cdot p^i \cdot (1-p)^{n-i}$}
\parbox{\columnwidth}{\centering $P(X > k) = 1 - P(X \leq k)$}
\parbox{\columnwidth}{\centering Eingabe Taschenrechner}
\parbox{\columnwidth}{\centering $\binom{n}{k} \, \widehat{=} \, n \, \vert nCr \vert \, k$}
\parbox{\columnwidth}{\centering Binomialverteilung kann mit der Normalverteilung approximiert werden, bedingung ist}
\parbox{\columnwidth}{\centering $X \sim B(n, p) \stackrel{a}{=} N(n \cdot p, n \cdot p \cdot (1-p))$}
\parbox{\columnwidth}{\centering falls gilt}
\parbox{\columnwidth}{\centering $n \cdot p \cdot (1-p) > 9$}

\framebox[\columnwidth][l]{Hypergeometrische Verteilung}
\parbox{\columnwidth}{\centering Ohne zurücklegen, Wahrscheinlichkeit ändert sich nach jedem Ereignis}
\parbox{\columnwidth}{\centering $X \sim H(N, M, n)$}
\parbox{\columnwidth}{\centering $n =:$ Stichprobenumfang}
\parbox{\columnwidth}{\centering $N =:$ Gesamtzahl}
\parbox{\columnwidth}{\centering $M =:$ Anzahl der Elemente mit der Eigenschaft}
\parbox{\columnwidth}{\centering $P(X = k) = \frac{\binom{M}{k} \cdot \binom{N-M}{n-k}}{\binom{N}{n}} $}
\parbox{\columnwidth}{\centering $P(X \leq k) = \sum \limits_{i=0}^k \frac{\binom{M}{i} \cdot \binom{N-M}{n-i}}{\binom{N}{n}} $}
\parbox{\columnwidth}{\centering $P(X > k) = 1 - P(X \leq k)$}

\framebox[\columnwidth][l]{Poisson Verteilung}
\parbox{\columnwidth}{\centering $X \sim Pois(\lambda)$}
\parbox{\columnwidth}{\centering $P(X = k) = \pi_\lambda(k) = \frac{\lambda^k}{k!} \cdot e^{-\lambda}$}

\framebox[\columnwidth][l]{Geometrische Verteilung}
\parbox{\columnwidth}{\centering $X \sim Geom(n, p)$}
\parbox{\columnwidth}{\centering $P(X = n) = (1-p)^{n-1} \cdot p^n$}
\parbox{\columnwidth}{\centering Beispiel: Ein Würfel wird so lange gewürfelt bis eine 6 Auftritt. Die Zufallsvariable X ist gleich Anzahl der Würfe}

\subsection{\framebox[\columnwidth][l]{Stetige Verteilungen}}

\framebox[\columnwidth][l]{Normalverteilung}
\parbox{\columnwidth}{\centering $X \sim N(\mu, \sigma^2)$}
\parbox{\columnwidth}{\centering Ist $X \sim N(0, 1)$ dann heißt sie Standardnormalverteilt}
\parbox{\columnwidth}{\centering Jede Normalverteilung kann standardisiert werden, das heißt die Mitte der Kurve wird auf den Nullpunkt gesetzt}
\parbox{\columnwidth}{\centering Wenn $X \sim N(\mu, \sigma^2)$ verteilt ist}
\parbox{\columnwidth}{\centering dann ist die standardisierte Zufallsvariable $Z = \frac{x-\mu}{\sigma} \sim N(0,1)$ standardnormalverteilt}
\parbox{\columnwidth}{\centering Ist die Zufallsvariable standardverteilt kann die Wahrscheinlichkeit aus der Tabelle abgelesen werden}
\parbox{\columnwidth}{\centering $P(X \leq k) = \Phi(k)$}
\parbox{\columnwidth}{\centering $P(X = k) = \Phi(k) = 0$}
\parbox{\columnwidth}{\centering $P(X \leq -k) = 1 - \Phi(+k)$}
\parbox{\columnwidth}{\centering allgemein gilt}
\parbox{\columnwidth}{\centering $X \sim N (\mu, \sigma^2)$}
\parbox{\columnwidth}{\centering $P(X \leq k) = \Phi(\frac{k-\mu}{\sigma})$}

\framebox[\columnwidth][l]{Quantile der Normalverteilung}
\parbox{\columnwidth}{\centering Tabelliert ist das $\beta-$Quantil $z_\beta$ der Normalverteilung $N(0,1)$}
\parbox{\columnwidth}{\centering $P(X \leq z_\beta) = \beta$}
\parbox{\columnwidth}{\centering $z_{1-\beta} = -z_\beta$}
\parbox{\columnwidth}{\centering Beispiel}
\parbox{\columnwidth}{\centering $\beta = 0.9 => z_\beta = 1.28155$}

\subsection{\framebox[\columnwidth][l]{Erwartungswert und Varianz}}
\subsubsection{\framebox[\columnwidth][l]{Erwartungswert}}

\framebox[\columnwidth][l]{Zufallsvariable mit diskreter Verteilung}
\parbox{\columnwidth}{\centering $\mu = E(X) = \sum \limits_{i=0}^n (x_i \cdot p_i)$}

\framebox[\columnwidth][l]{Zufallsvariable mit Dichtefunktion $f$}
\parbox{\columnwidth}{\centering $\mu = E(X) = \int_{-\infty}^{\infty} x \cdot f(x) dx $}

\framebox[\columnwidth][l]{Für Binomialverteilung}
\parbox{\columnwidth}{\centering $\mu = E(X) = n \cdot p$}

\framebox[\columnwidth][l]{Für geometrische Verteilung}
\parbox{\columnwidth}{\centering $\mu = E(X) = \frac{1}{p}$}

\framebox[\columnwidth][l]{Für Poissonverteilung}
\parbox{\columnwidth}{\centering $\mu = E(X) = \lambda$}

\framebox[\columnwidth][l]{Für Hypergeometrischeverteilung}
\parbox{\columnwidth}{\centering $E(S_n) = E(X_1 + ... + X_n) = n \cdot E(X_1) = n \cdot \frac{M}{N}$}

\framebox[\columnwidth][l]{Allgemeine Regeln für den Erwartungswert}
\parbox{\columnwidth}{\centering $a, b \in \mathbb{R}$}
\parbox{\columnwidth}{\centering $E(aX + b) = a \cdot E(X) + b$}
\parbox{\columnwidth}{\centering $E(X + Y) = E(X) + E(Y)$}
\parbox{\columnwidth}{\centering $E(aX + bY) = a \cdot E(X) + b \cdot E(Y)$}

\subsubsection{\framebox[\columnwidth][l]{Varianz}}

\framebox[\columnwidth][l]{Zufallsvariable mit diskreter Verteilung}
\parbox{\columnwidth}{\centering $\sigma^2 = Var(X) = \sum (x_i - \mu)^2 \cdot p_i$}

\framebox[\columnwidth][l]{Zufallsvariable mit Dichtefunktion $f$}
\parbox{\columnwidth}{\centering $Var(X) = E(X^2) - (E(X))^2$}

\framebox[\columnwidth][l]{Für Binomialverteilung}
\parbox{\columnwidth}{\centering $\sigma^2 = n \cdot p \cdot (1-p)$}

\framebox[\columnwidth][l]{Für geometrische Verteilung}
\parbox{\columnwidth}{\centering $\sigma^2 = \frac{1}{p^2} - \frac{1}{p}$}

\framebox[\columnwidth][l]{Für Poissonverteilung}
\parbox{\columnwidth}{\centering $\sigma^2 = Var(X) = E(X^2) - E(X)^2 = \lambda$}

\framebox[\columnwidth][l]{Für Hypergeometrischeverteilung}
\parbox{\columnwidth}{\centering $Var(S_n) = n \cdot \frac{M}{N} \cdot (1 - \frac{M}{N}) \cdot \frac{N-n}{N-1}$}

\framebox[\columnwidth][l]{Allgemeine Regeln für Varianz}
\parbox{\columnwidth}{\centering $Var(X + Y) = Var(X) + Var(Y) + 2 \cdot cov(X,Y)$}

\subsubsection{\framebox[\columnwidth][l]{Unabhängiger Zufallsvariablen}}

\framebox[\columnwidth][l]{Allgemeine Regeln}
\parbox{\columnwidth}{\centering $E(X \cdot Y = E(X) \cdot E(Y)$}
\parbox{\columnwidth}{\centering $Var(X + Y) = Var(X) + Var(Y)$}

\end{multicols*}
\end{document}